\makeatletter
\ifx\documentclass\@twoclasseserror\else
  \documentclass[a4j]{jarticle}
  \usepackage{bounddvi}
  \usepackage[utf,yoko]{endnotesj}
\fi
\makeatother
\begin{document}
日本史・日本文学等、縦組史・資料を素材とすることから論文も縦組で書かねばならない分野\endnote{ただし「歴史学」という括りで西洋・東洋史と混在の学術誌の場合、日本史であっても横組が「強要」されることもある。}にとって、従来、コンピュータでの論文執筆には様々な障碍があった。特に前近代文献の場合\endnote{近代文献であっても、史料における「見せ消ち」等の再現が困難であることは変わらない。}、独特の「割註」などは、通常のワープロソフトでの再現が難しいものである\endnote{「組文字」機能を使えば原理的には実現できないこともないが、長い「割註」の場合途方もない作業が要求され、現実的ではないだろう。}。そこで登場するのが\LaTeXe{} である。たとえば「割註」の場合、kunten2e.styという優れたスタイル・ファイル\endnote{大阪大学の金水敏氏によって提供されている。}によっていとも簡単に実現できる。

しかし日本語\LaTeXe{} はその組版能力においてはワープロなど及びもつかぬ柔軟性を有するものの、基本的にJIS 0218の範囲内の文字しか扱えないことが、数多くの漢字を必要とする古典籍の記述にとってはネックとなっていた。それを補うため、JISX 0212補助漢字を利用する方法なども開拓されてきた\endnote{一応私も、そのType1フォントを配布している。}ものの、これはあくまで「つなぎ」のものであり\endnote{たとえば、文字選定が「場当たり」的であるとして、規格そのものが低く評価されることも多いようである。}、フォント環境のユニコードへの移行が決定的となった状況において、日本語\LaTeXe{} 環境から膨大なグリフ数を誇るユニコード・フォントに簡単にアクセスできるようにする方法\endnote{将来的には、ユニコードに基礎を置いた多言語\TeX{} 環境であるOmegaへの移行が考えられるものの、これまで日本語\LaTeXe{} で蓄積されてきた様々なツールの移植は容易ではないため、直ちに移行することは困難だろう。}は、強く求められるものであったといえよう。

そういった状況のなか、ums.sty\endnote{稲垣淳氏による。}およびそのクロスプロットフォーム化\endnote{角藤亮氏が作成されたヴァーチャル・フォントによる恩恵が大きい。}を経て、それらの改良のうえに登場したのがutf.styである。齋藤修三郎氏によって作成されたこのパッケージでは、ums.styを引き継いでユニコード番号でグリフを呼び出せるのみならず、AdobeJapan 1-5のCIDコードによる指定を用い、ユニコードではアクセスできぬグリフまでをもフルに利用できるようになった\endnote{MacOSXの場合、標準でバンドルされているヒラギノ・フォントを利用することで、その機能を完全に利用できる。また、他の環境にあっても、AcrobatReaderのフォント・パック(小塚明朝)をインストールすることで、その殆どの機能の恩恵に与ることが可能だ。}。これにより、たとえば旧来の規格では「機種依存文字」であった丸囲み数字等が、\verb/\UTF{2460}/といったコマンドでプラットフォームを選ばず簡単に\UTF{2460}のように表示できるようになった。さらに現在、このutf.styの機能をさらに容易に利用するための様々なマクロ集が作成の途上にあり\endnote{これについては、齋藤氏とともに井上浩一氏のご尽力が大きいといえよう。}、その可能性は極めて大きく開かれているといえるだろう。かくいうこの文章も、注番号の表示にutf.styおよびそのマクロによる詰数字自動処理機能を利用することによって、従来のものにくらべ格段に美しい組版を実現できているものと思う。

\theendnotes
\end{document}
